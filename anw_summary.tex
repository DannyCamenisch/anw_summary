\documentclass[a4paper]{report}

\usepackage{amsmath}
\usepackage{tcolorbox} % for colorboxes
\usepackage{minted} % for code highlighting
\usepackage{xcolor} % for own colors

\usepackage[margin=4cm]{geometry} % custom page margin

\definecolor{dcGrayLight}{gray}{0.95}
\definecolor{dcOrange}{RGB}{255, 128, 0}
\definecolor{dcGreen}{RGB}{0, 154, 23}
\definecolor{dcBlue}{RGB}{0, 68, 129}
\definecolor{dcRed}{RGB}{161,0,14}
\definecolor{dcWhite}{RGB}{255, 255, 255}

\colorlet{dcRedLight}{red!5!white}

\setlength{\parindent}{0in} % remove indentation for paragraphs

\title{Algorithmen und Wahrscheinlichkeiten}
\author{Danny Camenisch (dcamenisch)}

\begin{document}
\maketitle
\tableofcontents


% --------------------------
%   Templates and Examples
% --------------------------

\chapter{Template}
\section{tcolorbox}

\begin{tcolorbox}[colback=dcWhite,colframe=dcOrange,title=\textbf{My Heading}]
    This is a \textbf{tcolorbox}.
\tcblower
    Here, you see the lower part of the box.
\end{tcolorbox}

\section{minted}

\begin{minted}[frame=lines, framesep=2mm, bgcolor=dcGrayLight, linenos]{java}
    // Hello.java
    import javax.swing.JApplet;
    import java.awt.Graphics;
    
    public class Hello extends JApplet {
        public void paintComponent(Graphics g) {
            g.drawString("Hello, world!", 65, 95);
        }    
    }
\end{minted}

% minted can also import from file like this:
% \inputminted{java}{main.java}


% --------------------------
%   Chapter 1.
% --------------------------

\chapter{Graphentheorie}
\section{Zusammenhang}

Wir erinnern uns an die Definition eines \textbf{zusammenhängenden} Graphen:

\begin{tcolorbox}[colframe=dcBlue,title=Definition]
    Ein Graph $G = (V,E)$ heisst \textbf{zusammenhängend}, wenn $\forall u,v \in V, u \neq v$ 
    ein Pfad von u nach v in $G$ existiert. \\

    $X$ heisst \textbf{u-v-Seperator}, wenn u und v in verschiedenen Zusammenhangskomponenten von
    $G[V \backslash X]$ liegen.
\end{tcolorbox}
\bigskip

Diese Definition besagt zwar ob ein Graph zusammenhängend ist oder nicht, man kann aber nichts darüber sagen,
wie stark ein Graph zusammenhängend ist. Dafür definieren wir sowohl Knoten- als auch Kantenzusammenhang:

\begin{tcolorbox}[colframe=dcBlue,title=Definition]
    Ein Graph $G = (V,E)$ heisst \textbf{k-zusammenhängend}, falls $|V| \geq k + 1$ und $\forall X \subseteq V$
    mit $|X| \geq k$ gilt: $G[V \backslash X]$ ist zusammenhängend. \\

    Ein Graph $G = (V,E)$ heisst \textbf{k-kanten-zusammenhängend}, falls $\forall X \subseteq E$
    mit $|X| \geq k$ gilt: $(V, E \backslash X)$ ist zusammenhängend.
\end{tcolorbox}
\bigskip

Ein Graph der 3-zusammenhängend ist, besitzt keine Seperator der Grösse 2 und ist dadurch auch 2-zusammenhängend. \\

Anschaulich zu merken: Wie viele Knoten oder Kanten muss man mindestens entfernen, bis der Graph nicht mehr
zusammenhängend ist.

\begin{tcolorbox}[colframe=dcGreen,title=Es Gilt]
    Knoten-Zusammenhang $\leq$ Kanten-Zusammenhang $\leq$ minimaler Knotengrad
\end{tcolorbox}
\bigskip

Ein (teils schwer zu beweisender) Satz erlaubt uns, eine äquivalente Definition von Zusammenhang
verwenden:

\begin{tcolorbox}[colframe=dcRed,title=Satz von Menger]
    Sei $G = (V, E)$ ein Graph. Dann gilt:
    \begin{enumerate}
        \item $G$ ist k-zusammenhängend $\Leftrightarrow \forall u,v \in V, u \neq v$ gibt es mindestens k intern-knotendisjunkte u-v-Pfade.
        \item $G$ ist k-kanten-zusammenhängend $\Leftrightarrow \forall u,v \in V, u \neq v$ gibt es mindestens k intern-kantendisjunkte u-v-Pfade.
    \end{enumerate}
\end{tcolorbox}
\bigskip

\section{Artikulationsknoten und Brücken}

\begin{tcolorbox}[colframe=dcBlue,title=Definition]
    Sei $G = (V,E)$ ein zusammenhängender Graph. Ein Knoten $v \in V$ heisst \textbf{Artikulationsknoten}
    (eng. cut vertex) genau dann wenn $G[V \backslash \{v\}]$ nicht zusammenhängend ist.
\end{tcolorbox}

\begin{tcolorbox}[colframe=dcBlue,title=Definition]
    Sei $G = (V,E)$ ein zusammenhängender Graph. Eine Kante $e \in E$ heisst \textbf{Brücke}
    (eng. cut edge) genau dann wenn $G[E \backslash \{e\}]$ nicht zusammenhängend ist.
\end{tcolorbox}

\begin{tcolorbox}[colframe=dcGreen,title=Es Gilt]
    Sei $G = (V,E)$ ein zusammenhängender Graph. Ist $\{x,y\} \in E$ eine Brücke so gilt für $x$
    (und analog auch für $y$): \\
    $$deg(x) = 1 \text{    oder    } x \text{ ist ein Artikulationsknoten}$$
\end{tcolorbox}
\bigskip

Die Umkehrung ist hier aber nicht immer wahr!

\begin{tcolorbox}[colframe=dcBlue,title=Definition]
    Sei $G = (V,E)$. Wir definieren eine Äquivalenzrelation auf $E$ wie folgt:
    $$e \sim f: \Leftrightarrow e = f \text{  oder  } \exists \text{ Kreis durch } e \text{ und } f$$
\end{tcolorbox}
\bigskip

Diese Äquivalenzklassen nennen wir auch \textbf{Blöcke}. Eine alternative Definition lautet wiefolgt:

\begin{tcolorbox}[colframe=dcBlue,title=Definition]
    Sei $G = (V,E)$ ein zusammenhängender Graph. Ein \textbf{Block} ist eine maximale Menge von Kanten, so
    dass je zwei dieser Kanten auf einem gemeinsamen Kreis liegen.
\end{tcolorbox}
\bigskip

Merke: Ein Block ist ein Subgraph, der 2-zusammenhängend ist.

\begin{tcolorbox}[colframe=dcGreen,title=Es Gilt]
    Zwei Blöcke schneiden sich - wenn überhaupt - immer in einem Artikulationsknoten.
\end{tcolorbox}
\bigskip

\section{Block-Graphen}

Wir erinnern uns an die Definition von bipartiten Graphen.

\begin{tcolorbox}[colframe=dcBlue,title=Definition]
    Ein Graph ist \textbf{bipartit}, wenn sich die Knotenmenge in zwei disjunkte Mengen
    $A$ und $B$ zerlegen lässt, sodass Kanten von $G$ nur zwischen $A$ und $B$ verlaufen.
    Wir verwenden dafür folgende Notation:
    $$V = A \uplus B$$
\end{tcolorbox}
\bigskip

Ein Block-Graph ist nun wiefolgt definitert:

\begin{tcolorbox}[colframe=dcBlue,title=Definition]
    Der Block-Graph von $G$ ist der bipartite Graph $T = (A \uplus B, E_T )$ mit
    \begin{itemize}
        \item $A =$ {Artikulationsknoten von $G$}
        \item $B =$ {Blöcke von $G$}
        \item $\forall a \in A , b \in B: \{a,b\} \in E_T \Leftrightarrow a$ inzident zu einer Kante in $b$
    \end{itemize}
\end{tcolorbox}
\bigskip

\end{document}