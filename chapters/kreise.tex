\chapter{Kreise}
\section{Eulertour}
\begin{definition}
    Sei $G = (V,E)$ ein Graph. Eine \textbf{Eulertour} ist ein geschlossener \text{Weg}, der jede Kante genau
    einmal enthält. Enthält ein Graph eine Eulertour, so nennt man ihn \textbf{eulersch}.
\end{definition}
\bigskip

Eine solche Tour lässt sich in $\mathcal{O}(|E|)$ finden - der Algorithmus (''schneller und langsamer Läufer'')
lässt sich ungefähr wie folgt beschreiben:

\begin{algorithm}
    \caption{EULERTOUR(G,s)}
    \begin{algorithmic}[1]
        \State $W \leftarrow$ \Call{RANDOMTOUR}{s} \Comment{''Läufer''}
        \State $v_{slow} \leftarrow $ Startknoten von $W$ \Comment{''Schildkröte''}
        \While{$v_{slow}$ ist nicht der letzte Knoten in $W$} 
            \State $v \leftarrow$ Nachfolger von $v_{slow}$ in $W$
            \If{$\exists (v,w) \in E, w \notin W$} \Comment{Ungenutze Kanten ab $v$}
                \State $W' \leftarrow$ \Call{RANDOMTOUR}{s}
                \State $W \leftarrow W_1 + W' + W_2$
            \EndIf
            \State $v_{slow} \leftarrow $ Nachfolger von $v_{slow}$ in $W$
        \EndWhile
        \Return $W$
    \end{algorithmic}
\end{algorithm}

\begin{algorithm}
    \caption{RANDOMTOUR(s)}
    \begin{algorithmic}[1]
        \State $v \leftarrow s$
        \State $W \leftarrow \{v\}$
        \While{$\exists (v,w) \in E, w \notin W$} \Comment{Ungenutze Kanten ab $v$}
            \State Wähle beliebigen Nachfolger $v_{next}$
            \State Hänge $v_{next}$ an $W$ analog
            \State $e \leftarrow \{v, v_{next}\}$
            \State Lösche $e$ aus $G$
            \State $v \leftarrow v_{next}$
        \EndWhile
        \State \Return $W$
    \end{algorithmic}
\end{algorithm}

Die Laufzeit ist leicht erklärt: Wir betrachten jede Kante genau einmal und ''löschen'' sie danach.

\begin{satz}[Satz]
    Ein zusammenhängender Graph $G = (V,E)$ enthält eine Eulertour $\Leftrightarrow$ der Grad jedes Knoten
    $v \in V$ ist gerade.
\end{satz}

\section{Hamiltonkreis}
\begin{definition}
    Sei $G = (V,E)$ ein Graph. Ein \textbf{Hamiltonkreis} ist ein Kreis, der alle Knoten von $V$ genau einmal
    durchläuft. Enthält ein Graph einen Hamiltonkreis, so nennt man ihn \textbf{hamiltonisch}.
\end{definition}
\bigskip

Ein klassisches Anwendungsbeispiel ist das Traveling Salesman Problem. Es wird vermutet, dass es keinen 
Algorithmus gibt, der in polynomieller Zeit bestimmt, ob es einen Hamiltonkreis in einem gegebenen Graphen 
gibt. \\

Es gibt aber einige Spezialfälle, für die es deutlich leichter ist, die Existenz eines Hamiltonkreises 
zu bestimmen:

\begin{itemize}
    \item Ein $n \times m$ Gitter enthält einen Hamiltonkreis genau dann wenn $n * m$ gerade ist
    \item Ein d-dimensionaler Hyperwürfel $H_d$ (Knotenmenge: $\{0, 1\}^d$, Kantenmenge: Alle Knotenpaare, welche sich in genau einer Koordinate unterscheiden) enthält einen Hamiltonkreis (auch für Dimensionen $d \geq 4$)
\end{itemize}





Wir bemerken den drastischen Laufzeitunterschied gegenüber der Eulertour: Dort benötigten wir gerade 
einmal $\mathcal{O}(|E|)$ Zeit, um eine solche zu finden, wären das Hamiltonkreisproblem NP-vollständig ist.