\chapter{Matching}

Viele Zuordnungsprobleme lassen sich als ein Graphenproblem modellieren. Nehmen wir etwa an, wir müssen 
Jobs auf Rechner verteilen, aber nicht alle Rechner können alle Jobs ausführen. Modellieren wir dies durch 
einen Graphen, wobei Jobs und Rechner durch Knoten dargestellt sind, und Kanten zwischen Job und Rechner 
bedeuten das Erfüllen der Vorraussetzungen für den Job. Nun suchen wir eine Zuordnung, die möglichst jedem 
Job einen Rechner und jedem Rechner einen Job (der ausgeführt werden kann) zuordnet.

\begin{definition}
    Eine Kantenmenge $M \subseteq E$ heisst \textbf{Matching} in einem Graphen $G = (V,E)$, falls kein
    Knoten des Graphen zu mehr als einer Kante aus $M$ inzident ist, oder formal ausgedrückt, wenn
    $$e \cap f = \emptyset \text{    für alle    } e,f \in M \text{    mit    } e \neq f$$
    Ein Knoten $v$ wir von $M$ \textbf{überdeckt}, falls eine Kante $e \in M$ gibt, die $v$ enthält. \\

    Ein Matching $M$ heisst \textbf{perfektes Matching}, wenn jeder Knoten durch genau eine Kante aus
    $M$ überdeckt wird, oder, anders ausgedrückt, wenn $|M| = \frac{|V|}{2} $.
\end{definition}
\bigskip

Wir intressieren uns oft dafür, ein möglichst grosses Matching zu finden. Hierfür müssen wir definieren, was ein
möglichst grosses Matching ist:

\begin{definition}
    Sei $G = (V,E)$ ein Graph mit einem Matching $M$.
    \begin{itemize}
        \item $M$ heisst \textbf{inklusionsmaximal}, wenn es kein Matching $M'$ mit $M \subseteq M'$ und $|M| < |M'|$ gibt.
        \item $M$ heisst \textbf{kardinalitätsmaximal} (oder nur maximal), wenn es kein Matching mit $|M| < |M'|$ gibt.
    \end{itemize}
    Ein kardinalitätsmaximales Matching ist auch immer inklusionsmaximal, aber nicht umgekehrt. Weiter gilt,
    $|M_{inc}| \geq |M_{max}| / 2$. Haben wir ein kardinalitätsmaximal Matching mit $M = |V| / 2$, überdeckt es also alle
    Knoten, so nenne wir es ein \textbf{perfektes Matching}.
\end{definition}

\section{Matching-Algorithmen}

Wir schauen uns nun ein paar Algorithmen an, mitwelchen wir Matchings finden können.

\begin{algorithm}
    \caption{Greedy-Matching(G)}
    \begin{algorithmic}[1]
        \State $M \leftarrow \emptyset$
        \While{$E \neq \emptyset$}
            \State wähle eine beliebige Kante $e \in E$
            \State $ M \leftarrow M \cup \{e\}$
            \State lösche $e$ und alle inzidenten Kanten in $G$
        \EndWhile
        \State \Return $M$
    \end{algorithmic}
\end{algorithm}

Mit dem Greedy-Algorithmus kann in $\mathcal{O}(|E|)$ ein Matching bestimmt werden. Es gilt
$|M_{Greedy}| \geq |M_{max}| / 2$, wobei $M_{max}$ ein kardinalitätsmaximales Matching ist.