\chapter{Matching}

Viele Zuordnungsprobleme lassen sich als ein Graphenproblem modellieren. Nehmen wir etwa an, wir müssen 
Jobs auf Rechner verteilen, aber nicht alle Rechner können alle Jobs ausführen. Modellieren wir dies durch 
einen Graphen, wobei Jobs und Rechner durch Knoten dargestellt sind, und Kanten zwischen Job und Rechner 
bedeuten das Erfüllen der Vorraussetzungen für den Job. Nun suchen wir eine Zuordnung, die möglichst jedem 
Job einen Rechner und jedem Rechner einen Job (der ausgeführt werden kann) zuordnet.

\begin{definition}
    Eine Kantenmenge $M \subseteq E$ heisst \textbf{Matching} in einem Graphen $G = (V,E)$, falls kein
    Knoten des Graphen zu mehr als einer Kante aus $M$ inzident ist, oder formal ausgedrückt, wenn
    $$e \cap f = \emptyset \text{    für alle    } e,f \in M \text{    mit    } e \neq f$$
    Ein Knoten $v$ wir von $M$ \textbf{überdeckt}, falls eine Kante $e \in M$ gibt, die $v$ enthält. \\

    Ein Matching $M$ heisst \textbf{perfektes Matching}, wenn jeder Knoten durch genau eine Kante aus
    $M$ überdeckt wird, oder, anders ausgedrückt, wenn $|M| = \frac{|V|}{2} $.
\end{definition}
\bigskip

Wir intressieren uns oft dafür, ein möglichst grosses Matching zu finden. Hierfür müssen wir definieren, was ein
möglichst grosses Matching ist:

\begin{definition}
    Sei $G = (V,E)$ ein Graph mit einem Matching $M$.
    \begin{itemize}
        \item $M$ heisst \textbf{inklusionsmaximal}, wenn es kein Matching $M'$ mit $M \subseteq M'$ und $|M| < |M'|$ gibt.
        \item $M$ heisst \textbf{kardinalitätsmaximal} (oder nur maximal), wenn es kein Matching mit $|M| < |M'|$ gibt.
    \end{itemize}
    Ein kardinalitätsmaximales Matching ist auch immer inklusionsmaximal, aber nicht umgekehrt. Weiter gilt,
    $|M_{inc}| \geq |M_{max}| / 2$. Haben wir ein kardinalitätsmaximal Matching mit $M = |V| / 2$, überdeckt es also alle
    Knoten, so nenne wir es ein \textbf{perfektes Matching}.
\end{definition}

\section{Matching-Algorithmen}

Wir schauen uns nun ein paar Algorithmen an, mitwelchen wir Matchings finden können.

\begin{algorithm}
    \caption{Greedy-Matching(G)}
    \begin{algorithmic}[1]
        \State $M \leftarrow \emptyset$
        \While{$E \neq \emptyset$}
            \State wähle eine beliebige Kante $e \in E$
            \State $ M \leftarrow M \cup \{e\}$
            \State lösche $e$ und alle inzidenten Kanten in $G$
        \EndWhile
        \State \Return $M$
    \end{algorithmic}
\end{algorithm}

Mit dem Greedy-Algorithmus kann in $\mathcal{O}(|E|)$ ein Matching bestimmt werden. Es gilt
$|M_{Greedy}| \geq |M_{max}| / 2$, wobei $M_{max}$ ein kardinalitätsmaximales Matching ist. \\

Solch ein Greedy-Matching muss nicht umbedingt ein kardinalitätsmaximales Matching sein. Wollen wir solche eines finden, dann
brauchen wir einen anderen Algorithmus. Dafür führen wir den Begriff des \textbf{augmentierenden Pfad} einführen.
\begin{definition}
    Für ein Matching $M$, nennen wir eine Pfad \textbf{$M$-augmentierender}, wenn er abwechselnd Kanten
    aus $M$ und nicht aus $M$ enthält, und in (unterschiedlichen) Knoten die nicht von $M$ überdeckt werden startet und endet.
\end{definition}
\bigskip

Wenn wir solch einen $M$-augmentierenden Pfad $P$ haben, können wir durch tauschen der Zugehörigkeit aller Kanten im Pfad das Matching
vergrössern. Wir notieren das tauschen/augmentieren als $M \oplus P$.

\begin{satz}[Satz von Berger]
    Jedes Matching, das nicht kardinalitätsmaximal ist, besitz einen augmentierenden Pfad.
\end{satz}
\bigskip

Nun können wir einen simplen Algorithmus verwenden, um solch ein grösseres Matching zu erhalten. Wir schauen
uns hier nur den Algorithmus für bipartite Graphen an.

\begin{algorithm}
    \caption{Bipartite-Matching(G)}
    \begin{algorithmic}[1]
        \State $M \leftarrow \emptyset$
        \Repeat
            \State $P \leftarrow$ \Call{Augmenting-Path}{G,M}
            \State $M \leftarrow M \oplus P$
        \Until $P = \emptyset$
        \State \Return $M$
    \end{algorithmic}
\end{algorithm}

\pagebreak
Das Finden des augmentierenden Pfades erreichen wir durch modifizierte Breitensuche:

\begin{algorithm}
    \caption{Augmenting-Path(G = $(A \uplus B, E)$, M)}
    \begin{algorithmic}[1]
        \State $L_0 \leftarrow $\{unüberdeckte Knoten aus $A$\} und markiere $L_0$ als besucht
        \For{i = 1...n}
            \If{i ungerade}
                \State $L_i \leftarrow $ \{unbesuchte Nachbaren von $L_{i-1}$ via Kanten in $E \setminus M$ \}
            \Else 
                \State $L_i \leftarrow $ \{unbesuchte Nachbaren von $L_{i-1}$ via Kanten in $M$ \}
            \EndIf
            \State markiere Knoten aus $L_i$ als besucht
            \If{ein Knoten $v$ in $L_i$ nicht überdeckt ist}
                \State \Return Pfad von $L_0$ nach $v$ (backtracking)
            \EndIf
        \EndFor
    \end{algorithmic}
\end{algorithm}

Da BFS in $\mathcal{O}(|V| + |E|)$ läuft, und unser Matching maximal eine Grösse von $\frac{n}{2} = \mathcal{O}(n)$
hat, läuft unser Algorithmus in $\mathcal{O}(|V| \cdot |E|)$. Es ist einfach zu sehen, dass der kürzeste Pfad von
$L_0$ nach $v$ die Länge $i$ hat. \\

Wir können diesen Algorithmus aber noch etwas optimieren, so dass er in $\mathcal{O}(\sqrt{|V|} \cdot (|V| + |E|))$ läuft. Anstatt den erst besten Pfad
zurückzugeben, suchen wir für alle unüberdeckten $v$ in $L_i$ knotendisjunkte Pfade nach $L_0$.

\begin{algorithm}
    \caption{Hopcroft-Karp(G)}
    \begin{algorithmic}[1]
        \State $M \leftarrow \emptyset$
        \Repeat
            \State $k \leftarrow$ länge eines kürzeste augmentierenden Pfades
            \State Finde alle disjunkte augmentierende Pfade $S$ der Länge $k$
            \For{Pfad P in S}
                \State $M \leftarrow M \oplus P$
            \EndFor
        \Until $P = \emptyset$
        \State \Return $M$
    \end{algorithmic}
\end{algorithm}
(\textit{Beweis siehe Vorlesung})