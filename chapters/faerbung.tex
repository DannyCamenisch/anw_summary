\chapter{Färbungen}

Viele Probleme in der Graphentheorie können wir durch eine Partition der Knotenmenge lösen. Ein
solches Problem ist zum Beispiel, dass erstellen eines Prüfungsplans - es wir eine Färbung gesucht,
so dass keine zwei benachbarten Knoten die gleiche Farbe haben.

\begin{definition}
    Eine \textbf{(Knoten-)Färbung} eines Graphen $G = (V, E)$ mit $k$ Farben ist eine Abbildung
    $c: V \to [k]$, so dass gilt

    $$c(u) \neq c(v) \text{ für alle Kanten } \{u,v\} \in E$$

    Die \textbf{chromatische Zahl} $\chi (G)$ ist die minimale Anzahl Farben, die für eine Färbung
    von $G$ benötigt wird.
\end{definition}
\bigskip

Ein klassisches Graphfärbungsproblem, welches einen herausragend komplizierten Beweis erfordert, ist
das 4-Farben-Theorem, welches besagt, dass jede Landkarte mit 4-Farben gefärbt werden kann, so dass 
jeweils benachbarte Länder unterschiedlich gefärbt sind. Im Allgemeinen gilt, dass ein planarer
Graph immer mit 4 Farben färbbar ist. \\

Graphen mit chromatischer Zahl $k$ nennt man auch $k$-partit, denn es gilt:

$$\chi (G) \leq k \Leftrightarrow G \text{ ist $k$-partit}$$

Besonders Interessant ist hier der Fall $k = 2$, also ein bipartiter Graph. Ob ein Graph bipartit ist,
lässt sich in $\mathcal{O}(|V| + |E|)$ bestimmen (DFS / BFS).

\begin{satz}[Satz]
    Für alle $k \geq 3$ ist das Problem,

    $$\text{Gegeben ein Graph $G = (V, E)$, gilt $\chi (G) \leq k$?}$$

    NP-vollständig.
\end{satz}
\bigskip

Färbbarkeit ist so schwer, da gilt $\forall k,r \in \mathbb{N}$ gibt es einen Graphen \textbf{ohne}
einen Kreis mit Länge $\leq k$, aber mit chromatischer Zahl $\geq r$. \\

Um eine Färbung zu finden, können wir wieder einen Greedy-Algorithmus benutzen.

\begin{algorithm}
    \caption{Greedy-Färbung(G)}
    \begin{algorithmic}[1]
        \State wähle eine beliebige Reihenfolge der Knoten in $V$
        \State $c[v_1] \leftarrow 1$
        \For{$i = 2$ to $i = n$}
            \State $c[v_i] \leftarrow$ min$\{ k \in \mathbb{N} | k \notin c(u) \text{ für alle } u \in N(v_i) \cap \{v_1, ... , v_{i-1}\} \}$
        \EndFor
    \end{algorithmic}
\end{algorithm}

Für jede Reihenfolge der Knoten, benötigt der Greedy-Algorithmus höchstens $\Delta (G) + 1$ viele Farben.
Jedoch gibt es eine Reihenfolge, so dass der Greedy-Algorithmus nur $\chi (G)$ viele Farben benötigt.
Mit etwas Heuristik kommen wir darauf, dass wenn wir die Knoten zuerst absteigen nach dem Knotengrad
ordnen, wir höchstens $k+1$ viele Farben brauchen. Dafür brauchen wir auch nur $\mathcal{O}(|E|)$ Zeit.
Wir wissen aber, dass wir dies nochmal auf $k$ viele Farben verbessern können.

\begin{satz}[Satz von Brooks]
    Einen zusammenhängenden Graphen $G$ für den gilt $G \neq K_n, G \neq C_{2n+1}$, kann in $\mathcal{O}(|E|)$
    Zeit eine Färbung mit $\Delta(G)$ vielen Farben gefunden werden.
\end{satz}
\textit{Für den genauen Algorithmus siehe Slides 8a} \bigskip

Damit können wir nun zeigen, wie wir in $\mathcal{O}(|E|)$ eine Färbung mit $\mathcal{O}(\sqrt{n})$ für 
einen 3-färbbaren Graphen finden können: Wir färben zuerst alle Knoten mit $\text{deg($v$) } \leq \sqrt{n}$ 
und deren Nachbarn. Dafür benötigen wir maximal drei Farben in jedem Schritt, also maximal $3\sqrt{n}$ 
Farben insgesamt. Wir entfernen daraufhin alle gefärbten Knoten. Sobald wir alle Knoten dementsprechend 
abgearbeitet haben, färben wir den Rest unseres Graphen mit dem Greedy-Algorithmus. Dafür benötigen wir 
$\sqrt{n}$ Farben, also total $\mathcal{O}(\sqrt{n})$.