\chapter{Artikulationsknoten und Brücken}

\begin{definition}
    Sei $G = (V,E)$ ein zusammenhängender Graph. Ein Knoten $v \in V$ heisst \textbf{Artikulationsknoten}
    (eng. cut vertex) genau dann wenn $G[V \backslash \{v\}]$ nicht zusammenhängend ist.
\end{definition}

\begin{definition}
    Sei $G = (V,E)$ ein zusammenhängender Graph. Eine Kante $e \in E$ heisst \textbf{Brücke}
    (eng. cut edge) genau dann wenn $G[E \backslash \{e\}]$ nicht zusammenhängend ist.
\end{definition}

\begin{lemma}
    Sei $G = (V,E)$ ein zusammenhängender Graph. Ist $\{x,y\} \in E$ eine Brücke so gilt für $x$
    (und analog auch für $y$): \\
    $$deg(x) = 1 \text{    oder    } x \text{ ist ein Artikulationsknoten}$$
\end{lemma}
\bigskip

Die Umkehrung ist hier aber nicht immer wahr! \\

Wie können wir nun solche Artikulationsknoten und Brücken finden? Dazu können
wir eine leicht modifizierte Version der bereits bekannten Tiefensuche verwenden. \\

Dazu müssen wir einige Beobachtungen machen. Zuerst, der von uns gewählte Startknoten $s$ der Tiefensuche,
und, für uns noch viel wichtiger, die Wurzel des Tiefensuchbaums $T$. Es wird schnell klar: Wenn $s$ im Tiefensuchbaum
eine Grad von mindestens 2 hat, so ist $s$ ein Artikulationsknoten. \\

Für jeden Knoten $v \neq s$ ist allerdings nicht so offensichtlich, ob $v$ ein Artikulationsknoten ist oder nicht.
Wieder können wir uns den Tiefensuchbaum zunutzen machen: $v$ ist ein Artikulationsknoten genau dann, wenn
$v$ im Baum Children (oder ganze Subtrees) besitzt, welche keine nicht-Baum-Kante zu einem Vorgänger von $v$ besitzen. \\

Um diese Voraussetzung zu überprüfen, definieren wir für unseren Graphen alle Kanten des Tiefensuchbaums als
\textit{Vorwärtskanten}, alle anderen als \textit{Rückwärtskanten}. Zusätzlich richten wir diese Kanten wie im Baum
(für Vorwärtskanten) bzw. vom höheren zum kleineren DFS Wert für Rückwärtskanten. Weiter definieren wir für jeden Knoten
$v \in V$: \\

low[v] $:=$ kleinste DFS-Nummer, die von $v$ aus durch einen Pfad mit beliebig vielen Vorwärtskanten und maximale
einer Rückwärtskante erreicht werden kann. \\

$
    \text{low}[v] = \text{ min} \left(
        \text{dfs}[v], \underset{(v,w) \in E}{\text{  min  }}
        \begin{cases}
            \text{dfs}[w], \text{ falls }(v,w) \text{ Restkante}\\
            \text{low}[w], \text{ falls }(v,w) \text{ Baumkante}
        \end{cases}
    \right)
$ \\
\\

Dann erhalten wir als Bedingung für Artikulationsknoten:

{\small 
\[
\text{v ist Artikulationsknoten} \Leftrightarrow
\begin{array}{l}
    v = s  v \text{ hat in } T \text{ Grad } \geq 2\\
    v \neq s \text{ und } \exists w \in V \text{ mit }\{v,w\} \in E(T) \wedge \text{ LOW[w]} \geq \text{DFS[v]}
\end{array}
\]
}

Wenden wir unser Wissen über Brücken an, so können wir diese im gleichen Zug berechnen: Eine gerichtete Kante 
${v, w}$ des Tiefensuchbaums (nur diese kommen in Frage) ist eine Brücke genau dann wenn low[$w$] $>$ dfs[$v$].
Restkanten sind niemals Brücken.\\

Unser angepasster DFS-Algorithmus sieht dann wie folgt aus, wobei als Input der Graph $G$ 
und der Startknoten $s$ verlangt wird:

\begin{algorithm}
    \caption{DFS(G,s)}
    \begin{algorithmic}[1]
        \State $\forall v \in V : \text{dfs}[v] \leftarrow 0$
        \State num $\leftarrow 0$ \Comment{Anzahl besuchter Knoten}
        \State $T \leftarrow \emptyset$ \Comment{Menge der Kanten im Tiefensuchbaum}
        \State \Call{DFS-Visit}{G,s}
        \If{\text{s hat in T mindestens Grad 2}}
            \State isArtVert[$s$] $\leftarrow$ TRUE
        \Else
        \State isArtVert[$s$] $\leftarrow$ FALSE
        \EndIf
    \end{algorithmic}
\end{algorithm}
\pagebreak
\begin{algorithm}
    \caption{DFS-Visit(G,v)}
    \begin{algorithmic}[1]
        \State num $\leftarrow$ num $ + 1$
        \State dfs[$v$] $\leftarrow$ num
        \State low[$v$] $\leftarrow$ dfs[$v$]
        \State isArtVert[$v$] $\leftarrow$ FALSE

        \For{$(v,w) \in E$}
            \If{dfs[$w$] $= 0$}
                \State $T \leftarrow T + (v,w)$
                \State val $\leftarrow$ \Call{DFS-Visit}{G,w} \Comment{low-Wert des direkten Nachfolgers}
                \If{val $\geq$ dfs[$v$]}
                \State isArtVert[$v$] $\leftarrow$ TRUE
                    \If{val $>$ dfs[$v$]}
                        \State isBridge[$v$][$w$] $\leftarrow$ TRUE
                    \EndIf
                \EndIf
                \State low[$v$] $\leftarrow$ min\{low[$v$], val\}
            \Else \Comment{Also ist dfs[$w$] $\neq 0$}
                \State low[$v$] $\leftarrow$ min\{low[$v$], dfs[$w$]\}
            \EndIf
        \EndFor

        \State \Return low[$v$]
    \end{algorithmic}
\end{algorithm}


\begin{satz}[Satz]
    In einem zusammenhängenden Graphen $G = (V,E)$, der als Adjazenzlist gespeichert ist, lassen sich
    alle Artikulationsknoten und Brücken in Zeit $\mathcal{O}(|E|)$ berechnen.
\end{satz}
\bigskip