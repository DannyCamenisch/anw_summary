\chapter{Grundlagen}

Wahrscheinlichkeiten spielen eine wichtige Rolle in der Informatik. Viele Algorithmen beruhen auf
Zufall (z.Bsp. QuickSort oder Hashing) und jegliche Art von Kryptographie währe ohne Stochastik 
unmöglich. In diesem ersten Kapitel der Wahrscheinlichkeiten definieren wir ein paar grundlegende 
Begriffe und Notationen.

\begin{definition}
    Ein \textbf{diskreter Wahrscheinlichkeitsraum} ist durch eine endliche oder zumindest abzählbare
    \textbf{Ereignismenge} $\Omega = \{ \omega_1, \omega_2, ...  \}$ von \textbf{Elementarereignissen}. 
    Jedem Elementarereignis $\omega_i$ ist eine (Elementar)-Wahrscheinlichkeiten Pr$[\omega_i]$
    zugeordnet. Dabei gilt $0 \leq$  Pr$[\omega_i] \leq 1$ und 

    $$\sum_{\omega \in \Omega}^{} \text{Pr} [\omega] = 1$$

    Eine Menge $E \subseteq \Omega$ heisst \textbf{Ereignis}. Die Wahrscheinlichkeiten eines Ereignisses
    ist definiert durch die Summe seiner Elementarereignissen. Weiter bezeichnen wir mit $\bar{E} = \Omega \backslash E$
    das \textbf{Komplementärereignis} zu $E$.
\end{definition}
\bigskip

Aus der Definition des Wahrscheinlichkeitsraums folgen einige triviale, aber sehr nützliche 
Konsequenzen:

\begin{satz}[Satz]
    Für Ereignisse $A_1, B_1, A_2, B_2, ...$ gilt:
    \begin{itemize}
        \item Pr$[\emptyset] = 0$, Pr$[\Omega] = 1$
        \item $0 \leq \text{Pr}[A] \leq 1$
        \item Pr$[\bar{A}] = 1 - \text{Pr}[A]$
        \item $A \subseteq B \Rightarrow \text{Pr}[A] \leq \text{Pr}[B]$
    \end{itemize}
\end{satz}
\bigskip

Ausserdem lässt sich der erste wichtige Satz für den Umgang mit mehreren Ereignissen formulieren:

\begin{satz}[Additionssatz]
    Falls die Ereignisse $A_1, A_2, ... , A_n$ \text{paarweise disjunkt} sind, so folgt:

    $$\text{Pr}[ \bigcup_{i=1}^{n} A_i] = \sum_{i = 1}^{n} \text{Pr}[A_i]$$
\end{satz}
\bigskip

Dies gilt nur für disjunkte Ereignisse. Im Allgemeinen gilt:

\begin{satz}[Boolesche Ungleichung / Union Bound]
    Für Ereignisse Ereignisse $A_1, A_2, ... , A_n$ gilt:

    $$\text{Pr}[ \bigcup_{i=1}^{n} A_i] \leq \sum_{i = 1}^{n} \text{Pr}[A_i]$$
\end{satz}
\bigskip

Wir können für die genaue Berechnung der Vereinigung von Ereignissen die \textbf{Siebformel} anwenden.
Wie legen wir die Wahrscheinlichkeiten für Elementarereignissen sinnvoll fest? Generell gilt das
\textbf{Prinzip von Laplace}, d.h. wenn nichts dagegen spricht, gehen wir davon aus, dass alle 
Elementarereignissen die gleich Wahrscheinlichkeit haben. Ein Wahrscheinlichkeitsraum, indem alle
Elementarereignissen die gleiche Wahrscheinlichkeit haben, wird deshalb auch \textbf{Laplace-Raum}
genannt. Als Beispiel für einen Laplace-Raum ist der Würfelwurf oder das Ziehen einer Karte aus einem
Kartenspiel zu betrachten.


An dieser Stelle lohnt es sich, den \textbf{Binomialkoeffizienten} einzuführen. Dies ist die Kurzfassung
für: 

\begin{quote}
    Wie viele Möglichkeiten gibt es, aus $n$ Elementen $k$ Elemente (ohne zurücklegen) auszuwählen?
\end{quote}

Den Binomialkoeffizienten schreiben wir als 
$$\binom{n}{k} = \frac{n!}{k! \cdot (n-k)!}$$

Im Allgemeinen sind folgende vier Formeln sehr hilfreich zu wissen:

\begin{center}
    \begin{tabular}[h]{c || c | c }
        & geordnet & ungeordnet \\
       \hline
       \hline
       mit Zurücklegen & $n^k$ & $\binom{n + k - 1}{k}$  \\
       \hline
       ohne Zurücklegen & $n^{\underline{k}}$ & $\binom{n}{k}$ \\
       \hline
   \end{tabular}
\end{center}
